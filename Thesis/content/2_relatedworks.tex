\chapter{Related works}
\label{chap:related works}

There are already pretty much researches about anomaly detection, some of them referred to streaming data and online learning. In this section, we list some widely used classical machine learning-based approaches, as well as some autoencoder based researches, and finally we refer the provious work on neural network incremental learning.

\section{Classical machine learning based approaches}
\label{sec:Classical machine learning based approaches}

As an important component of data mining and machine learning, anomaly detection has been investigated using plenty efficient models.\\\\
\textbf{LOF model} \\

In anomaly detection, the Local Outlier Factor (LOF) is a common density-based approach. LOF shares some concepts with DBSCAN such as ‘core distance’ and ‘reachability distance’, in order to estimate local density. Here, points with substantially lower local density than their neighbors are considered as anomalies. LOF shows competitive performance in many anomaly detection tasks, especially when dealing with data with unevenly density distribution. However, when getting a numerical factor from LOF model, it is actually hard to define a threshold automatically for the judgement of anomaly.\\\\

\textbf{OCSVM model} \\

Another widely used model is the domain based One-class Support Vector Machine. As an unsupervised one-class classifier, OCSVM takes only normal data as input, and generates a decision surface to separate them from the anomaly states.  By analyzing anomalies, the datasets are always bias to the normal part, and anomaly appear only rarely. So, this kind of one-class classifiers avoid making balance between the two classes. Besides, they also take advantage of classical support vector machine, with the help of kernel method, they can also deal with linearly not separable data.
Although classical machine learning approaches can handle most of the normal anomaly detection, only few of those approaches could be directly or after some modification used for time series or streaming data, while they ignore the temporal dependency between samples.



\section{Autoencoder-based anomaly detection approaches}
\label{sec:Autoencoder-based anomaly detection approaches}

LSTMs-Autoencoders are originally widely used for text generation. Text data are usually embedded into vector as input of autoencoder. And the tasks are either generate temporal relevant text on the decoder side or learn text representation in the hidden layer \cite{phraserepresentation}. As text data are relevant in the sense of words within a sentence or between sentences. It is similar to the streaming data temporal dependency problem.\\

Sutskever et al. \cite{seq2seq} use a deep LSTMs-based sequence to sequence model for language translation. In their work, the deep LSTMs encoder take single sentence as input, and learn a hidden vector of a fixed dimensionality, and then a different LSTMs decoder decodes it to the target sentence.  As a translation task, they found that this encoder-decoder architecture can capture long sentences and sensible phrases, especially they achieved better performance with deep LSTMs in compare with shallow LSTMs. In addition, a valuable found is, reversing the order of words in the input sentence makes the optimization problem much easier and achieved better performance. The LSTMs based model outperforms non-LSTMs model on the long input sentence cases (more than 35 words) since its long-term memory ability.\\

Li et al. \cite{hierarchicalseq2seq} did similar research on long paragraph text or even entire document generation using LSTMs-autoencoders. Their main contribute is the hierarchical sentences representation. The model learns words level, sentence level, paragraph level and document level information with each respectively a LSTMs layer, so that the model captures very long-term temporal information. Moreover, they introduced an attention based hierarchical sequence to sequence model that connect the most relative part between encoder and decoder like the works around a final punctuation. They experiment with documents over 100 words, the results show that hierarchical and attention-based hierarchical LSTMs learns even better long-term temporal information than standard LSTMs-encoder-decoder models.\\

As autoencoders achieves great successes in text data and speech processing, they are also used on time series anomaly detection in terms of temporal dependently data. These models train autoencoders with only normal data, and anomaly data as unknown patterns. Then the autoencoder can only reconstruct normal patterns, large reconstruction error indicates anomaly. An early work \cite{eps} uses the vanilla autoencoder to detect abnormal status of the electric power system. In order to capture temporal information, they applied sliding window on the raw data as input. As anomaly scoring method, they evaluated each sliding window with respect to their reconstruction error. As some measures in the autoencoder output vectors are more sensible to anomalies than others, they use the average absolute deviation of reconstruction error as anomaly score. And the anomaly threshold is chosen by large amount of experiments over normal data.\\

An important reason of using autoencoder for anomaly detection is its ability of dealing with high-dimensional. Sakurada et al. \cite{ dimensionalityreduction} experimented with time series data that consist of 10-100 variables with no linear correlation. Comparing with reconstruction using PCA or Kernel PCA techniques, using the autoencoder reconstruction error is more easily to recognize anomalies.\\

In further researches, Malhotra et al. \cite{lstmad}\cite{encdecad} develop the application of LSTMs-autoencoder in sequence learning into anomaly detection problem. They proposed stacked LSTM networks model to learn high level temporal patterns. The show that LSTMs outperforms normal RNNs based anomaly detection model and avoid facing to the gradient vanishing problem. They also detect anomaly based on the reconstruction error. The scoring function is based on the parameters of a estimated normal distribution of a validation set. Their experiments show that the model performs good in variety kinds of datasets. A variation of this model \cite{timenet} has been shown that achieves better performance in the anomaly detection tasks. The author tells that, using a constant as input of decoder instead of read time series value improves the performance of model.


\section{Online incremental learning with autoencoders}
\label{sec:Online incremental learning with autoencoders}

Zhou et al. proposed an online incremental updating method for denoising autoencoders by modifying the hidden layer neurons in order to deal with the non-stationary streaming data properties. The kern ideal are two steps, merging hidden layer neurons if there are information redundancy, and adding hidden layer neurons to capture new knowledge. Their experimental result shows comparable or better reconstruction result than non-incremental approaches with only few data used during initialization. And they show that their incremental feature learning methods performs more adaptively and robustly to highly non-stationary input distribution.

Dong et al proposed a 2-step anomaly detection mechanism with incremental autoencoders. The implemented the system with ensembled autoencoders in multithreads to leverage parallel computing when large volumes of data arrive. Besides their 2-step mechanism check anomaly in the first step and verify anomaly data with previous and subsequent data (to differ between anomalous state and concept drift) to reduce false-positive rate in anomaly detection. In the experimental results, they show that their model outperforms commonly used tree-based anomaly detection model especially when concept drift presents and speed up the online processing speed with mini-batch learning and online learning in multithreads.







